\documentclass[12pt]{article}
\usepackage[utf8]{inputenc}
\usepackage{amsmath}
\usepackage{graphicx}
\usepackage{geometry}
\usepackage{siunitx}
\usepackage{float}
\geometry{margin=1in}
\usepackage[french]{babel}
\usepackage{listings}
\usepackage{xcolor}
\lstset{
  language=Python,
  basicstyle=\ttfamily\small,
  keywordstyle=\color{blue},
  stringstyle=\color{red},
  commentstyle=\color{gray},
  showstringspaces=false,
  breaklines=true,
  frame=single,
  extendedchars=false, % important pour que 'literate' fonctionne bien
  literate=
    {é}{{\'e}}1 {è}{{\`e}}1 {ê}{{\^e}}1 {ë}{{\"e}}1
    {à}{{\`a}}1 {â}{{\^a}}1 {ä}{{\"a}}1
    {ç}{{\c{c}}}1
    {ù}{{\`u}}1 {û}{{\^u}}1 {ü}{{\"u}}1
    {î}{{\^i}}1 {ï}{{\"i}}1
    {ô}{{\^o}}1 {ö}{{\"o}}1
}

\sisetup{output-decimal-marker = {,}, per-mode=symbol, group-separator = {\,}}

\title{Étude d'une centrale thermique solaire}
\author{Réponses aux questions}
\date{\today}

\begin{document}

\maketitle

\section*{Partie I : Étude du récepteur solaire}

\subsection*{1.1}
Les différentes raisons expliquant l'efficacité de \SI{70}{\percent} entre l'énergie solaire reçue par le récepteur et l'énergie solaire incidente sur les héliostats sont :
\begin{itemize}
  \item Réflexions imparfaites et diffusion par les miroirs ;
  \item Pertes atmosphériques (absorption et diffusion) ;
  \item Mauvais alignement ou suivi solaire imparfait ;
  \item Salissures ou vieillissement des héliostats.
\end{itemize}

\subsection*{1.2}
La puissance reçue par le récepteur depuis les miroirs est :
\begin{align*}
P_{\text{récepteur}} &= S_{\text{hel}} \cdot E_s \cdot \eta_m \\
&= \SI{75000}{m^2} \cdot \SI{1000}{W.m^{-2}} \cdot 0{,}70 \\
&= \boxed{\SI{52.5}{MW}}
\end{align*}

\subsection*{2.1}
En posant $\lambda_1=\SI{1}{\mu m}, \lambda_2=\SI{6}{\mu m}, \lambda_3=\SI{16}{\mu m}$ et $\epsilon_1=0,98, \epsilon_2 = 0,92, \epsilon_3 = 0,9, \epsilon_4 = 0,75$, on a
\begin{align*}
\epsilon(T) &= \alpha(T) \\ &= \int_{0}^{+\infty} \epsilon_{\lambda}(\lambda) f_{\lambda}(T, \lambda) \, d\lambda \\ &= \epsilon_1 \cdot F_{0-\lambda_{1} T} + \epsilon_2 \cdot (F_{0-\lambda_{2} T} - F_{0-\lambda_{1} T}) + \epsilon_3 \cdot (F_{0-\lambda_{3} T} - F_{0-\lambda_{2} T}) + \epsilon_4 \cdot (1 - F_{0-\lambda_{3} T})
\end{align*}
Pour $T=547K$ on obtient $\boxed{\epsilon=0,8890}$ et pour $T=5760$, on obtient $\boxed{\alpha=0,9628}$ (en utilisant la table de fraction d'énergie du corps noir fournie en annexe).

\subsection*{2.2}
Le facteur de forme \( F_{\text{e} \rightarrow \text{r}} \) de l'environnement vers le recepteur vaut trivialement 1, car l'intégralité du flux du rectangle rouge est capté. Par réciprocité :
\[
F_{\text{r} \rightarrow \text{e}} = \frac{S_e}{S_r} = \boxed{0{,}65}
\]

\subsection*{2.3}
On a :
\[
Ra = Gr \cdot Pr
\quad \text{avec} \quad 
\Delta T = \SI{254}{K},\quad L = \SI{10}{m},\quad T_m = \SI{420}{K},\quad \beta = \frac{1}{T}
\]
Aux conditions données :
\[
Pr = 0{,}7,\quad \mu = \SI{2,4e-5}{kg.m^{-1}.s^{-1}},\quad \rho = \SI{0,83}{kg.m^{-3}}, \quad \lambda = \SI{0,035}{W.m^{-1}.K^{-1}}
\]
Ainsi $Ra > 10^9$ et on utilise la corrélation :
\[
Nu_L = 0{,}13 \cdot (Gr_L \cdot Pr)^{0{,}33}
\]
Soit 
\begin{align*}
h &= 0{,}13 \cdot \frac{\lambda}{L} \cdot (Gr_L \cdot Pr)^{0{,}33} \\
&= \boxed{\SI{7,04}{W.m^{-2}.K^{-1}}}
\end{align*}

\subsection*{3.1}
La puissance absorbée après \( n \) réflexions est (démonstration par récurrence triviale):
\[
\alpha S_{\text{hel}} \eta_m E_s \cdot (\rho F_{r \rightarrow r})^n = \alpha S_{\text{hel}} \eta_m E_s \cdot \left[(1 - \alpha)(1 - F_{r \rightarrow e})\right]^n
\]
En sommant sur \( n \), on obtient :
\begin{align*}
P_{\text{abs}} &= \frac{\alpha}{1 - \alpha \cdot F_{r \rightarrow e}} \cdot S_{\text{hel}} \cdot \eta_m \cdot E_s \\
&= \frac{0{,}96}{1 - 0{,}96 \cdot 0{,}65} \cdot \SI{52.5e6}{W} \\
&= \boxed{\SI{133,89}{MW}}
\end{align*}

\subsection*{3.2}
Avec l'analogie électrique :
\[
\phi^{\text{nette}}_{r \rightarrow e} \cdot R = M^0_e - M^0_r = \sigma \left(T_a^4 - T_r^4\right)
\]
où :
\[
R = \frac{1 - \epsilon}{\epsilon S} + \frac{1}{S F_{r \rightarrow e}}
\]
D'où :
\[
\phi^{\text{nette}}_{r \rightarrow e} = \frac{\sigma S (T_a^4 - T_r^4)}{\frac{1 - \epsilon}{\epsilon} + \frac{1}{F_{r \rightarrow e}}}
\]

\subsection*{3.3}
On a :
\[
\dot{m} = \frac{\SI{66000}{kg.h^{-1}}}{\SI{3600}{s.h^{-1}}} = \SI{18.33}{kg.s^{-1}}, \quad
\Delta h = \SI{2689}{kJ.kg^{-1}} = \SI{2.689e6}{J.kg^{-1}}
\]

Ainsi la puissance transmise au fluide est:
\begin{align*}
P_{\text{fluide}} &= \dot{m} \cdot \Delta h \\ &= \SI{18.33}{kg.s^{-1}} \cdot \SI{2.689e6}{J.kg^{-1}} \\ &= \boxed{\SI{49.3}{MW}}
\end{align*}

\subsection*{3.4}
En régime permanent on a:
\[
0 = P_{abs} - \phi^{nette}_{r \rightarrow e} - P_{\text{fluide}} + \phi_{convection}
\]
Soit:
\[
P_{\text{fluide}} - P_{abs}  = -\phi^{nette}_{r \rightarrow e} + \phi_{convection}
\]

En posant
\[
A = P_{\text{fluide}} - P_{abs} \quad B = -\frac{\sigma S}{\frac{1-\epsilon}{\epsilon} + \frac{1}{F_{r \rightarrow e}}} \quad C = 2hS
\]
Cela revient à résoudre
\[
A = B(T_a^4 - T_r^4) + C(T_a - T_r)
\]
En utilisant le script python suivant qui implémente la méthode des itérations sur l'équation linéarisé, il vient $T_r=...$. (il n'y a pas de solution à l'equation proposée... (cf graphique du programme python) je pense que y'a une erreur qlq part, a voir plus tard.

\lstinputlisting{resolution_iterations.py}

\section*{Partie III : Étude thermodynamique}

\textbf{Caluler les points du cycle}
\begin{table}[h!]
\centering
\begin{tabular}{|c|c|c|c|c|c|}
\hline
\textbf{Point} & \textbf{Température (°C)} & \textbf{Pression (bar)} & \textbf{Entropie massique (kJ/kg·K)} & \textbf{Enthalpie massique (kJ/kg)} & \textbf{Titre vapeur} \\
\hline
1 & 280 & 40 & 6260 & 2902 & 1 \\
\hline
2 &  & 0,123 & 6950 & 2228 &  \\
\hline
3 & 50 & 0,123 &  & 209 & 0 \\
\hline
4 &  & 40 &  &  & 0 \\
\hline
\end{tabular}
\caption{Propriétés thermodynamiques aux 4 points du cycle}
\label{tab:points_thermo}
\end{table}

calcul valeurs point 1 par coolprop
calcul rendement point 2, puis coolprop
point 3 par coolprop, car dans le prolongement
point 4 calcul rendement

\textbf{Les tracer dans un diagramme T-S}
insérer photo

\textbf{Calculer la puissance mécanique récupérable pour la production électrique}

\textbf{Calculer le rendement de cycle}
puissance récupérable / puissance fournie

\end{document}

