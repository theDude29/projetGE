\documentclass[12pt]{article}
\usepackage[utf8]{inputenc}
\usepackage{amsmath}
\usepackage{graphicx}
\usepackage{geometry}
\usepackage{siunitx}
\usepackage{float}
\geometry{margin=1in}
\usepackage[french]{babel}
\usepackage{listings}
\usepackage{xcolor}
\usepackage{tabularx}
\lstset{
  language=Python,
  basicstyle=\ttfamily\small,
  keywordstyle=\color{blue},
  stringstyle=\color{red},
  commentstyle=\color{gray},
  showstringspaces=false,
  breaklines=true,
  frame=single,
  extendedchars=false, % important pour que 'literate' fonctionne bien
  literate=
    {é}{{\'e}}1 {è}{{\`e}}1 {ê}{{\^e}}1 {ë}{{\"e}}1
    {à}{{\`a}}1 {â}{{\^a}}1 {ä}{{\"a}}1
    {ç}{{\c{c}}}1
    {ù}{{\`u}}1 {û}{{\^u}}1 {ü}{{\"u}}1
    {î}{{\^i}}1 {ï}{{\"i}}1
    {ô}{{\^o}}1 {ö}{{\"o}}1
}

\sisetup{output-decimal-marker = {,}, per-mode=symbol, group-separator = {\,}}

\title{Étude d'une centrale thermique solaire}
\author{Réponses aux questions}
\date{\today}

\begin{document}

\maketitle

\section*{Partie I : Étude du récepteur solaire}

\subsection*{1.1}
Les différentes raisons expliquant l'efficacité de \SI{70}{\percent} entre l'énergie solaire reçue par le récepteur et l'énergie solaire incidente sur les héliostats sont :
\begin{itemize}
  \item Réflexions imparfaites et diffusion par les miroirs ;
  \item Pertes atmosphériques (absorption et diffusion) ;
  \item Mauvais alignement ou suivi solaire imparfait ;
  \item Salissures ou vieillissement des héliostats.
\end{itemize}

\subsection*{1.2}
La puissance reçue par le récepteur depuis les miroirs est :
\begin{align*}
P_{\text{récepteur}} &= S_{\text{hel}} \cdot E_s \cdot \eta_m \\
&= \SI{75000}{m^2} \cdot \SI{1000}{W.m^{-2}} \cdot 0{,}70 \\
&= \boxed{\SI{52.5}{MW}}
\end{align*}

\subsection*{2.1}
En posant $\lambda_1=\SI{1}{\mu m}, \lambda_2=\SI{6}{\mu m}, \lambda_3=\SI{16}{\mu m}$ et $\epsilon_1=0,98, \epsilon_2 = 0,92, \epsilon_3 = 0,9, \epsilon_4 = 0,75$, on a
\begin{align*}
\epsilon(T) &= \alpha(T) \\ &= \int_{0}^{+\infty} \epsilon_{\lambda}(\lambda) f_{\lambda}(T, \lambda) \, d\lambda \\ &= \epsilon_1 \cdot F_{0-\lambda_{1} T} + \epsilon_2 \cdot (F_{0-\lambda_{2} T} - F_{0-\lambda_{1} T}) + \epsilon_3 \cdot (F_{0-\lambda_{3} T} - F_{0-\lambda_{2} T}) + \epsilon_4 \cdot (1 - F_{0-\lambda_{3} T})
\end{align*}
Pour $T=547K$ on obtient $\boxed{\epsilon=0,8890}$ et pour $T=5760$, on obtient $\boxed{\alpha=0,9628}$ (en utilisant la table de fraction d'énergie du corps noir fournie en annexe).

\subsection*{2.2}
Le facteur de forme \( F_{\text{e} \rightarrow \text{r}} \) de l'environnement vers le recepteur vaut trivialement 1, car l'intégralité du flux du rectangle rouge est capté. Par réciprocité :
\[
F_{\text{r} \rightarrow \text{e}} = \frac{S_e}{S_r} = \boxed{0{,}65}
\]

\subsection*{2.3}
On a :
\[
Ra = Gr \cdot Pr
\quad \text{avec} \quad 
\Delta T = \SI{254}{K},\quad L = \SI{10}{m},\quad T_m = \SI{420}{K},\quad \beta = \frac{1}{T}
\]
Aux conditions données :
\[
Pr = 0{,}7,\quad \mu = \SI{2,4e-5}{kg.m^{-1}.s^{-1}},\quad \rho = \SI{0,83}{kg.m^{-3}}, \quad \lambda = \SI{0,035}{W.m^{-1}.K^{-1}}
\]
Ainsi $Ra > 10^9$ et on utilise la corrélation :
\[
Nu_L = 0{,}13 \cdot (Gr_L \cdot Pr)^{0{,}33}
\]
Soit 
\begin{align*}
h &= 0{,}13 \cdot \frac{\lambda}{L} \cdot (Gr_L \cdot Pr)^{0{,}33} \\
&= \boxed{\SI{7,04}{W.m^{-2}.K^{-1}}}
\end{align*}

\subsection*{3.1}
La puissance absorbée après \( n \) réflexions est (démonstration par récurrence triviale):
\[
\alpha S_{\text{hel}} \eta_m E_s \cdot (\rho F_{r \rightarrow r})^n = \alpha S_{\text{hel}} \eta_m E_s \cdot \left[(1 - \alpha)(1 - F_{r \rightarrow e})\right]^n
\]
En sommant sur \( n \), on obtient :
\begin{align*}
P_{\text{abs}} &= \frac{\alpha}{1 - \alpha \cdot F_{r \rightarrow e}} \cdot S_{\text{hel}} \cdot \eta_m \cdot E_s \\
&= \frac{0{,}96}{1 - 0{,}96 \cdot 0{,}65} \cdot \SI{52.5e6}{W} \\
&= \boxed{\SI{133,89}{MW}}
\end{align*}

\subsection*{3.2}
Avec l'analogie électrique :
\[
\phi^{\text{nette}}_{r \rightarrow e} \cdot R = M^0_e - M^0_r = \sigma \left(T_a^4 - T_r^4\right)
\]
où :
\[
R = \frac{1 - \epsilon}{\epsilon S} + \frac{1}{S F_{r \rightarrow e}}
\]
D'où :
\[
\phi^{\text{nette}}_{r \rightarrow e} = \frac{\sigma S (T_a^4 - T_r^4)}{\frac{1 - \epsilon}{\epsilon} + \frac{1}{F_{r \rightarrow e}}}
\]

\subsection*{3.3}
On a :
\[
\dot{m} = \frac{\SI{66000}{kg.h^{-1}}}{\SI{3600}{s.h^{-1}}} = \SI{18.33}{kg.s^{-1}}, \quad
\Delta h = \SI{2689}{kJ.kg^{-1}} = \SI{2.689e6}{J.kg^{-1}}
\]

Ainsi la puissance transmise au fluide est:
\begin{align*}
P_{\text{fluide}} &= \dot{m} \cdot \Delta h \\ &= \SI{18.33}{kg.s^{-1}} \cdot \SI{2.689e6}{J.kg^{-1}} \\ &= \boxed{\SI{49.3}{MW}}
\end{align*}

\subsection*{3.4}
En régime permanent on a:
\[
0 = P_{abs} - \phi^{nette}_{r \rightarrow e} - P_{\text{fluide}} + \phi_{convection}
\]
Soit:
\[
P_{\text{fluide}} - P_{abs}  = -\phi^{nette}_{r \rightarrow e} + \phi_{convection}
\]

En posant
\[
A = P_{\text{fluide}} - P_{abs} \quad B = -\frac{\sigma S}{\frac{1-\epsilon}{\epsilon} + \frac{1}{F_{r \rightarrow e}}} \quad C = 2hS
\]
Cela revient à résoudre
\[
A = B(T_a^4 - T_r^4) + C(T_a - T_r)
\]
En utilisant le script python suivant qui implémente la méthode des itérations sur l'équation linéarisé, il vient $T_r=...$. (il n'y a pas de solution à l'equation proposée... (cf graphique du programme python) je pense que y'a une erreur qlq part, a voir plus tard.

\lstinputlisting{resolution_iterations.py}

\section*{Partie III : Étude thermodynamique}

\textbf{Caluler les points du cycle}
\\
Nous renseignons toutes les informations sur les différents points dans le tableau suivant, les explications sont données plus bas.

\begin{table}[h!]
\centering
\begin{tabularx}{\textwidth}{|c|X|X|X|X|}
\hline
\textbf{Point} & \textbf{Température (°C)} & \textbf{Pression (bar)} & \textbf{Entropie massique (kJ/kg·K)} & \textbf{Enthalpie massique (kJ/kg)} \\
\hline
1 & 280 & 40 & 6260 & 2902 \\
\hline
2is & 50 & 0.123 & 6260 & 2004 \\
\hline
2 & 50 & 0.123 & 6950 & 2228 \\
\hline
3 & 50 & 0.123 & 704 & 209 \\
\hline
4is & - & 40 & 704 & 214 \\
\hline
4 & 50.6 & 40 & 710 & 215.25 \\
\hline
\end{tabularx}
\caption{Propriétés thermodynamiques aux 4 points du cycle}
\label{tab:points_thermo}
\end{table}

\textit{Point 1}
\\
La pression et la température sont données sur le schéma, on peut donc déduire l'entropie et l'enthalpie massique à l'aide de Coolprop.
De plus, on sait qu'il s'agit purement de vapeur à ce point là.
\\

\textit{Point 2}
\\
Pour obtenir le point 2, on détermine tout d'abord l'enthalpie du point 2is dans le cas d'une dilatation isentropique puis on utilise le rendement pour déterminer la vraie valeur de l'enthalpie ainsi que les autres informations sur ce point.
\\
La pression doit être la même qu'au point 3 (0.123 bar), et elle nous est donnée, car la transformation dans le condenseur sera isobare.
\\
Dans le domaine L+V, isobare implique isotherme donc la température sera aussi la même que le point 3, c'est à dire 50°C.
\\
En intersectant les droites verticale correspondant à une transforamtion isentropique de 1 à 2is et horizontale à une transforamtion isotherme de 2 à 3 on obtient ainsi le point 2is et donc la valeur de l'enthalpie dans le cas isentropique (voir schéma plus bas).
\\
On peut alors calculer la valeur de l'enthalpie massique du point 2 :

\begin{align*}
\eta &= \frac{\Delta h_{reel}}{\Delta h_{is}} \\
\Rightarrow h_2 &= h_1 + \eta(h_{2,is}-h_1) \\
&= 2902 + 0.75\cdot(2004-2902) \\
&= 2228 kJ/kg
\end{align*}
Une fois que l'on a ça, on peut déduire l'entropie massique à l'aide de Coolprop et on obtient 6950 kJ/kg·K.
\\

\textit{Point 3}
\\
Pour ce point, on sait qu'il se situe à 50°C et 0.123 bar sur la courbe de saturation, donc on peut déterminer toutes ses propriétés directement avec Coolprop sans faire de calculs.
\\

\textit{Point 4}
\\
On applique un raisonnement similaire à celui du point 2, mais cette fois le rendement se défini de manière inverse car c'est une pompe et non une turbine.
\\
On obtient le point 4is en intersectant la droite verticale à 704 kJ/kg·K avec l'isobare à 40 bar car la transformation 4 $\rightarrow$ 1 est isobare.
\\
On détermine ensuite le point 4 avec le calcul suivant :

\begin{align*}
\eta &= \frac{\Delta h_{is}}{\Delta h_{reel}} \\
\Rightarrow h_4 &= h_3 + \frac{h_{4,is}-h_3}{\eta} \\
&= 209 + \frac{214-209}{0.8} \\
&= 215.25 kJ/kg
\end{align*}
On peut maintenant déduire la température et l'entropie massique au point 4 pour le tracer, toujours à l'aide de Coolprop.
\\

\textbf{Les tracer dans un diagramme T-S}
insérer scan
\\

\textbf{Calculer la puissance mécanique récupérable pour la production électrique}
\\
Puissance récupérable = - Puissance partant dans la turbine
\begin{align*}
P_{turbine} &= - D(h_2-h_1) = D(h_1-h_2)\\
&= 66000kg/h\cdot(2902-2228)kJ/kg \\
&= \frac{66000}{3600} \cdot 674 W \\
&= 12,4 kW
\end{align*}
\\

\textbf{Calculer le rendement de cycle}
\\
Rendement = Puissance récupérable / Puissance fournie
\begin{align*}
\eta &= \displaystyle\left\lvert\frac{h_2-h_1}{h_1-h_4}\displaystyle\right\rvert\\
&= \frac{2228-2902}{2902-215} = \frac{674}{2687} \\
\eta &= 0.25
\end{align*}
\end{document}

